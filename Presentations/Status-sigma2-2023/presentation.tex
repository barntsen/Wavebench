% Template for presentation
\documentclass[xcolor=dvipsnames]{beamer}
\usecolortheme[named=Brown]{structure}   %Brown color scheme
\setbeamertemplate{navigation symbols}{} %No navigation clutter
\usepackage{tikz}
\usetikzlibrary{arrows,decorations.pathmorphing,backgrounds,positioning,fit}
\usetikzlibrary{datavisualization.formats.functions}
\usetikzlibrary{patterns}
%%     
%Here are some macro's saving time and labour:     
%     
%\newcommand{\const}{\mbox{const}}      
%\newcommand{\est}{\mbox{{\tiny est}}}      
%\newcommand{\im}{\mbox{$\Im \mbox{m}$}}      
%\newcommand{\obs}{\mbox{{\tiny obs}}}      
%\newcommand{\otherwise}{\mbox{otherwise}}      
%\newcommand{\real}{\mbox{$\Re \mbox{e}$}}      
%\newcommand{\sign}{\mbox{sign}}      
%\newcommand{\sinc}{\mbox{sinc}}      
%
%\renewcommand{\p}{\mbox{$\partial$}}      
%\renewcommand{\d}{\mbox{$\partial$}}      
%\newcommand{\w}{\mbox{$\omega$}}      
%
\newcommand{\AAA}{\mbox{\boldmath $A$}}   
\newcommand{\BB}{\mbox{\boldmath $B$}}     
\newcommand{\CC}{\mbox{\boldmath $C$}}     
\newcommand{\DD}{\mbox{\boldmath $D$}}     
\newcommand{\EE}{\mbox{\boldmath $E$}}     
\newcommand{\FF}{\mbox{\boldmath $F$}}   
\newcommand{\GG}{\mbox{\boldmath $G$}}   
\newcommand{\HH}{\mbox{\boldmath $H$}}   
\newcommand{\II}{\mbox{\boldmath $I$}}   
\newcommand{\JJ}{\mbox{\boldmath $J$}}   
\newcommand{\KK}{\mbox{\boldmath $K$}}   
\newcommand{\LL}{\mbox{\boldmath $L$}}   
\newcommand{\MM}{\mbox{\boldmath $M$}}   
\newcommand{\NN}{\mbox{\boldmath $N$}}   
\newcommand{\OO}{\mbox{\boldmath $O$}}   
\newcommand{\PP}{\mbox{\boldmath $P$}}   
\newcommand{\QQ}{\mbox{\boldmath $Q$}}   
\newcommand{\RR}{\mbox{\boldmath $R$}}   
\newcommand{\SSS}{\mbox{\boldmath $S$}}   
\newcommand{\TT}{\mbox{\boldmath $T$}}   
\newcommand{\UU}{\mbox{\boldmath $U$}}   
\newcommand{\VV}{\mbox{\boldmath $V$}}   
\newcommand{\WW}{\mbox{\boldmath $W$}}   
\newcommand{\XX}{\mbox{\boldmath $X$}}   
\newcommand{\YY}{\mbox{\boldmath $Y$}}   
\newcommand{\ZZ}{\mbox{\boldmath $Z$}}   
%
\newcommand{\aaa}{\mbox{\boldmath $a$}}     
\newcommand{\bb}{\mbox{\boldmath $b$}}     
\newcommand{\cc}{\mbox{\boldmath $c$}}     
\newcommand{\dd}{\mbox{\boldmath $d$}}     
\newcommand{\ee}{\mbox{\boldmath $e$}}   
\newcommand{\ff}{\mbox{\boldmath $f$}}   
%\newcommand{\ggg}{\mbox{\boldmath $g$}}   
\newcommand{\hh}{\mbox{\boldmath $h$}}   
\newcommand{\ii}{\mbox{\boldmath $i$}}   
\newcommand{\jj}{\mbox{\boldmath $j$}}   
\newcommand{\kk}{\mbox{\boldmath $k$}}   
%\newcommand{\lll}{\mbox{\boldmath $l$}}   
\newcommand{\mm}{\mbox{\boldmath $m$}}   
\newcommand{\nn}{\mbox{\boldmath $n$}}   
\newcommand{\pp}{\mbox{\boldmath $p$}}   
\newcommand{\qq}{\mbox{\boldmath $q$}}   
\newcommand{\rr}{\mbox{\boldmath $r$}}   
\newcommand{\sss}{\mbox{\boldmath $s$}}   
\newcommand{\ttt}{\mbox{\boldmath $t$}}   
\newcommand{\uu}{\mbox{\boldmath $u$}}   
\newcommand{\vv}{\mbox{\boldmath $v$}}   
\newcommand{\ww}{\mbox{\boldmath $w$}}   
\newcommand{\xx}{\mbox{\boldmath $x$}}   
\newcommand{\yy}{\mbox{\boldmath $y$}}   
\newcommand{\zz}{\mbox{\boldmath $z$}}   
%
\newcommand{\balpha}{\mbox{\boldmath $\alpha$}}     
\newcommand{\bpsi}{\mbox{\boldmath $\psi$}}     
\newcommand{\bphi}{\mbox{\boldmath $\phi$}}     
\newcommand{\bbeta}{\mbox{\boldmath $\beta$}}     
\newcommand{\btheta}{\mbox{\boldmath $\theta$}}     
\newcommand{\bdelta}{\mbox{\boldmath $\delta$}}     
\newcommand{\bgamma}{\mbox{\boldmath $d$}}     
\newcommand{\bGamma}{\mbox{\boldmath $\Gamma$}}     
\newcommand{\bLambda}{\mbox{\boldmath $\Lambda$}}     
\newcommand{\bmu}{\mbox{\boldmath $\mu$}}     
\newcommand{\bnabla}{\mbox{\boldmath $\nabla$}}     
\newcommand{\brho}{\mbox{\boldmath $\rho$}}     
\newcommand{\bSigma}{\mbox{\boldmath $\Sigma$}}     
\newcommand{\bsigma}{\mbox{\boldmath $\sigma$}}     
\newcommand{\bxi}{\mbox{\boldmath $\xi$}}     
\newcommand{\bepsilon}{\mbox{\boldmath $\epsilon$}}     
\newcommand{\blambda}{\mbox{\boldmath $\lambda$}}     
\newcommand{\BLambda}{\mbox{\boldmath $\Lambda$}}     
%-------------------------------------%
%  \Appendix - a new appendix command %
%-------------------------------------%
%The appendix command is used as in
% \Appendix{A}{The wave equation as a matrix equation}
\newcommand {\Appendix}[2]{
              \section*{APPENDIX #1: #2}
              \setcounter{equation}{0}
              \renewcommand{\theequation} 
              {A-\arabic{equation}}}
\newcommand {\Appendices}[2]{
              \section*{APPENDIX #1: #2 }
              \setcounter{equation}{0}
              \renewcommand{\theequation} 
              {#1-\arabic{equation}}}
%------------------------------------%
%    \aref - a new cite command.     % 
%------------------------------------%
\newcommand{\aref}[2]{\nocite{#1}#2} 
%----------------------------------------
%\eqref -an equation reference command
%----------------------------------------
%\newcommand{\eqref}[1]{(\ref{#1})}
%\newcommand{\eqref}[1]{\ref{#1}}

\usepackage{graphicx}
\usepackage{natbib}
\usepackage{multimedia}
\begin{document}
%     
%Here are some macro's saving time and labour:     
%     
%\newcommand{\const}{\mbox{const}}      
%\newcommand{\est}{\mbox{{\tiny est}}}      
%\newcommand{\im}{\mbox{$\Im \mbox{m}$}}      
%\newcommand{\obs}{\mbox{{\tiny obs}}}      
%\newcommand{\otherwise}{\mbox{otherwise}}      
%\newcommand{\real}{\mbox{$\Re \mbox{e}$}}      
%\newcommand{\sign}{\mbox{sign}}      
%\newcommand{\sinc}{\mbox{sinc}}      
%
%\renewcommand{\p}{\mbox{$\partial$}}      
%\renewcommand{\d}{\mbox{$\partial$}}      
%\newcommand{\w}{\mbox{$\omega$}}      
%
\newcommand{\AAA}{\mbox{\boldmath $A$}}   
\newcommand{\BB}{\mbox{\boldmath $B$}}     
\newcommand{\CC}{\mbox{\boldmath $C$}}     
\newcommand{\DD}{\mbox{\boldmath $D$}}     
\newcommand{\EE}{\mbox{\boldmath $E$}}     
\newcommand{\FF}{\mbox{\boldmath $F$}}   
\newcommand{\GG}{\mbox{\boldmath $G$}}   
\newcommand{\HH}{\mbox{\boldmath $H$}}   
\newcommand{\II}{\mbox{\boldmath $I$}}   
\newcommand{\JJ}{\mbox{\boldmath $J$}}   
\newcommand{\KK}{\mbox{\boldmath $K$}}   
\newcommand{\LL}{\mbox{\boldmath $L$}}   
\newcommand{\MM}{\mbox{\boldmath $M$}}   
\newcommand{\NN}{\mbox{\boldmath $N$}}   
\newcommand{\OO}{\mbox{\boldmath $O$}}   
\newcommand{\PP}{\mbox{\boldmath $P$}}   
\newcommand{\QQ}{\mbox{\boldmath $Q$}}   
\newcommand{\RR}{\mbox{\boldmath $R$}}   
\newcommand{\SSS}{\mbox{\boldmath $S$}}   
\newcommand{\TT}{\mbox{\boldmath $T$}}   
\newcommand{\UU}{\mbox{\boldmath $U$}}   
\newcommand{\VV}{\mbox{\boldmath $V$}}   
\newcommand{\WW}{\mbox{\boldmath $W$}}   
\newcommand{\XX}{\mbox{\boldmath $X$}}   
\newcommand{\YY}{\mbox{\boldmath $Y$}}   
\newcommand{\ZZ}{\mbox{\boldmath $Z$}}   
%
\newcommand{\aaa}{\mbox{\boldmath $a$}}     
\newcommand{\bb}{\mbox{\boldmath $b$}}     
\newcommand{\cc}{\mbox{\boldmath $c$}}     
\newcommand{\dd}{\mbox{\boldmath $d$}}     
\newcommand{\ee}{\mbox{\boldmath $e$}}   
\newcommand{\ff}{\mbox{\boldmath $f$}}   
%\newcommand{\ggg}{\mbox{\boldmath $g$}}   
\newcommand{\hh}{\mbox{\boldmath $h$}}   
\newcommand{\ii}{\mbox{\boldmath $i$}}   
\newcommand{\jj}{\mbox{\boldmath $j$}}   
\newcommand{\kk}{\mbox{\boldmath $k$}}   
%\newcommand{\lll}{\mbox{\boldmath $l$}}   
\newcommand{\mm}{\mbox{\boldmath $m$}}   
\newcommand{\nn}{\mbox{\boldmath $n$}}   
\newcommand{\pp}{\mbox{\boldmath $p$}}   
\newcommand{\qq}{\mbox{\boldmath $q$}}   
\newcommand{\rr}{\mbox{\boldmath $r$}}   
\newcommand{\sss}{\mbox{\boldmath $s$}}   
\newcommand{\ttt}{\mbox{\boldmath $t$}}   
\newcommand{\uu}{\mbox{\boldmath $u$}}   
\newcommand{\vv}{\mbox{\boldmath $v$}}   
\newcommand{\ww}{\mbox{\boldmath $w$}}   
\newcommand{\xx}{\mbox{\boldmath $x$}}   
\newcommand{\yy}{\mbox{\boldmath $y$}}   
\newcommand{\zz}{\mbox{\boldmath $z$}}   
%
\newcommand{\balpha}{\mbox{\boldmath $\alpha$}}     
\newcommand{\bpsi}{\mbox{\boldmath $\psi$}}     
\newcommand{\bphi}{\mbox{\boldmath $\phi$}}     
\newcommand{\bbeta}{\mbox{\boldmath $\beta$}}     
\newcommand{\btheta}{\mbox{\boldmath $\theta$}}     
\newcommand{\bdelta}{\mbox{\boldmath $\delta$}}     
\newcommand{\bgamma}{\mbox{\boldmath $d$}}     
\newcommand{\bGamma}{\mbox{\boldmath $\Gamma$}}     
\newcommand{\bLambda}{\mbox{\boldmath $\Lambda$}}     
\newcommand{\bmu}{\mbox{\boldmath $\mu$}}     
\newcommand{\bnabla}{\mbox{\boldmath $\nabla$}}     
\newcommand{\brho}{\mbox{\boldmath $\rho$}}     
\newcommand{\bSigma}{\mbox{\boldmath $\Sigma$}}     
\newcommand{\bsigma}{\mbox{\boldmath $\sigma$}}     
\newcommand{\bxi}{\mbox{\boldmath $\xi$}}     
\newcommand{\bepsilon}{\mbox{\boldmath $\epsilon$}}     
\newcommand{\blambda}{\mbox{\boldmath $\lambda$}}     
\newcommand{\BLambda}{\mbox{\boldmath $\Lambda$}}     
%-------------------------------------%
%  \Appendix - a new appendix command %
%-------------------------------------%
%The appendix command is used as in
% \Appendix{A}{The wave equation as a matrix equation}
\newcommand {\Appendix}[2]{
              \section*{APPENDIX #1: #2}
              \setcounter{equation}{0}
              \renewcommand{\theequation} 
              {A-\arabic{equation}}}
\newcommand {\Appendices}[2]{
              \section*{APPENDIX #1: #2 }
              \setcounter{equation}{0}
              \renewcommand{\theequation} 
              {#1-\arabic{equation}}}
%------------------------------------%
%    \aref - a new cite command.     % 
%------------------------------------%
\newcommand{\aref}[2]{\nocite{#1}#2} 
%----------------------------------------
%\eqref -an equation reference command
%----------------------------------------
%\newcommand{\eqref}[1]{(\ref{#1})}
%\newcommand{\eqref}[1]{\ref{#1}}

%==============================================================================
\title{Status HPC projects}
\author{B. Arntsen } 
\institute[NTNU]{
  NTNU\\
  Department of Geoscience and petroleum \\
  \texttt{borge.arntsen@ntnu.no}
}
\date{October 2023}
\begin{frame}
 \titlepage
\end{frame}
%==============================================
%---------------------------------------------------
\begin{frame}{HPC Projects}
%--------------------------------------------------
\begin{enumerate}
  \item Simulation of ship noise (EU) start 2024
  \item Inversion of the earths crust in the pacific 
  \item Inversion of the Mantel (North Atlantic)
\end{enumerate}  
\end{frame}
%---------------------------------------------------
\begin{frame}{EU SEASOUNDS}
%--------------------------------------------------
\begin{itemize}
  \item Large scale simulation of noise propagation
	in isfjorden, Svalbard.
  \item Software: SPECFEM 3D public domain
  \item OpenCL, do not expect problems using the LUMI GPU cluster
  \item Recruiting PhD, expect start in Q3.
\end{itemize}
\end{frame}
%---------------------------------------------------
\begin{frame}{Inversion of pacific crust}
%--------------------------------------------------
\begin{columns}
\column{0.5\textwidth}
\begin{itemize}
  \item Rockseis software			
  \item C++ 120k+ codelines
  \item Port to multiple GPU architectures
  \item Single source + Out of bounds check etc..
  \item Language/Compilers with GPU accelleration
	      and standard language
  \item Domain Specific Language
  \item Unified memory
  \item Initial port of mini application
\end{itemize}
\column{0.5\textwidth}
\includegraphics[width=0.5\textwidth]{Fig_fwi_4Hz_OK}
\includegraphics[width=0.5\textwidth]{map}
\end{columns}
\end{frame}
%----------------------------------------------
\begin{frame}{Mini application}
%-----------------------------------------------
\begin{center}
\movie[height = 0.6\textwidth, width = 0.8\textwidth, showcontrols] {Video}{fig-1.mp4}
\end{center} 
\end{frame}
%----------------------------------------------
\begin{frame}{Mini application}
%-----------------------------------------------
\includegraphics[width=1.0\textwidth]{walltime-geforce}
\end{frame}
%----------------------------------------------
\begin{frame}{Mini application}
%-----------------------------------------------
\includegraphics[width=1.0\textwidth]{walltime-a}
\end{frame}
%----------------------------------------------
\begin{frame}{Mini application}
%-----------------------------------------------
\includegraphics[width=1.0\textwidth]{walltime}
\end{frame}
%----------------------------------------------
\begin{frame}{Rock Seis GPU port}
%-----------------------------------------------
\begin{itemize}
  \item Time critical parts ported to GPU
  \item Single source DSL generates C, CUDA or HIP
	using Unified Memory
  \item Reasonable speedup vs cpu on A100
  \item Too slow on Mi250x ?
\end{itemize}
\end{frame}
%----------------------------------------------
\begin{frame}{Future code development}
%-----------------------------------------------
\begin{itemize}
  \item Python 
  + Standard Language or DSL with support
	for GPU acceleration and 
  \item  Julia ?
\end{itemize}
\end{frame}
\end{document}


